\documentclass[UTF8,a4paper,11pt,twocolumn]{ctexart}

% 导入必要的宏包
\usepackage{geometry}
\usepackage{setspace}
\usepackage{graphicx}
\usepackage{cite}
\usepackage{hyperref}
\usepackage{booktabs}
\usepackage{amsmath}
\usepackage{amssymb}
\usepackage{array}
\usepackage{fancyhdr}
\usepackage{caption}
\usepackage{subcaption}
\usepackage{changepage}
\usepackage{tabularx}
\usepackage{adjustbox}
\usepackage{multirow}
\usepackage{tikz}
\usetikzlibrary{shapes,arrows,positioning,calc}

% 收紧浮动体与表格间距
\setlength{\abovecaptionskip}{4pt}
\setlength{\belowcaptionskip}{4pt}
\setlength{\textfloatsep}{6pt}
\setlength{\floatsep}{6pt}
\setlength{\intextsep}{6pt}
\setlength{\dbltextfloatsep}{6pt}
\setlength{\dblfloatsep}{6pt}

% 表格行高与列间距微调
\renewcommand{\arraystretch}{0.94}
\setlength{\tabcolsep}{6pt}

% 页面设置
\geometry{left=1.8cm,right=1.8cm,top=2.5cm,bottom=2.5cm,columnsep=0.6cm}

% 设置单倍行间距
\setstretch{1.0}

% 超链接设置
\hypersetup{
    colorlinks=true,
    linkcolor=black,
    citecolor=black,
    urlcolor=blue
}

% 段落首行缩进
\setlength{\parindent}{2em}

% 中文字体设置
\xeCJKsetup{AutoFakeSlant=true}  % 自动生成伪斜体

% 页眉页脚设置
\pagestyle{fancy}
\fancyhf{}
\fancyhead{}
\fancyfoot[C]{\thepage}
\renewcommand{\headrulewidth}{0pt}
\renewcommand{\footrulewidth}{0pt}

% 首页特殊样式
\fancypagestyle{firstpage}{
    \fancyhf{}
    \fancyhead[R]{\small}
    \fancyfoot[C]{\thepage}
    \renewcommand{\headrulewidth}{0pt}
    \renewcommand{\footrulewidth}{0pt}
}

% 章节标题格式设置
\ctexset{
    section={
        format=\zihao{4}\heiti\raggedright,
        number=\arabic{section},
        name={,},
        beforeskip=1ex,
        afterskip=0.5ex
    },
    subsection={
        format=\zihao{5}\heiti\raggedright,
        number=\arabic{section}.\arabic{subsection},
        name={,},
        beforeskip=0.5ex,
        afterskip=0.5ex
    },
    subsubsection={
        format=\hspace{2em}\zihao{5}\kaishu\raggedright,
        number=\arabic{section}.\arabic{subsection}.\arabic{subsubsection},
        name={,},
        beforeskip=0.5ex,
        afterskip=0.5ex
    }
}

% 图表标题设置
\captionsetup{
    labelfont=bf,
    textfont=normalfont,
    justification=centering,
    font=small
}
\captionsetup[figure]{
    font={\zihao{-5}\songti},
    position=bottom,
    labelsep=space
}
\captionsetup[table]{
    font={\zihao{-5}\songti},
    position=top,
    labelsep=space
}

% 表注格式设置
\newcommand{\tablenote}[1]{\vspace{0.2em}\par\noindent{\zihao{-5}\rmfamily #1}}

% 表格内容字体大小
\newcommand{\tablesize}{\small}

\begin{document}

% 使用twocolumn[]使标题、作者、摘要跨双栏显示
\twocolumn[
\begin{@twocolumnfalse}
% 中文标题(小2号宋体加粗,居中)
\begin{center}
\zihao{-2}\songti\textbf{正交学习:知识蒸馏与注意力的互补偏置}

\vspace{0.3cm}
% 中文作者(5号楷体,居中)
\zihao{5}\kaishu
罗小娟$^{1}$,左博文$^{1}$

\vspace{0.2cm}
% 中文单位(5号楷体,居中)
(1. 华东理工大学 信息科学与工程学院,上海 200237)
\end{center}

\vspace{0.3cm}

% 中文摘要
\begin{adjustwidth}{2em}{2em}
\hspace{2em}\zihao{5}\heiti 摘要:\zihao{5}\kaishu 深度卷积神经网络在图像分类任务中表现优异,但由于计算成本高昂,难以部署在移动设备上。本文系统地研究了轻量级注意力机制与知识蒸馏在移动优先架构上对细粒度视觉识别的互补效应。研究表明,将高效注意力模块(ECA和SimAM)与知识蒸馏相结合,能够使MobileNetV3在计算开销几乎可忽略的情况下实现显著的性能提升。在Food-101数据集上,本文框架达到了78.50\%的准确率,相比基线(74.23\%)提升了4.27个百分点,同时仅保持5.5M参数量和0.22 GFLOPs计算量。通过全面的消融实验,本文揭示了注意力机制与知识蒸馏具有互补效应:注意力引导特征选择,而蒸馏提供更丰富的监督信号。本文系统比较了六种注意力机制,并提供了关于参数高效设计(ECA、SimAM)为何优于重量级机制的实证洞见。在Oxford Flowers-102数据集上的跨域验证实验表明,该方法在不同细粒度识别任务上均能实现一致的相对改进(+2.33\%),证明了方法的领域无关性。从理论视角,本文提出"正交学习"框架,首次将互补效应解释为两种正交归纳偏置的协同:知识蒸馏实现架构归纳偏置的跨模型转移,注意力机制引入特征级注意力偏置,二者功能正交使得增益可加性叠加。本工作为资源受限应用中优化效率-准确率权衡提供了方法论指导。

\hspace{2em}\zihao{5}\heiti 关键词:\zihao{5}\kaishu 轻量级神经网络;知识蒸馏;注意力机制;细粒度图像分类;移动端部署
\end{adjustwidth}

{\zihao{5}\heiti
\begin{minipage}[t]{0.48\textwidth}
\hspace{4em}中图分类号:TP391.41
\end{minipage}
\hfill
\begin{minipage}[t]{0.48\textwidth}
\raggedright
文献标志码:A
\end{minipage}
}

\vspace{0.5cm}
\end{@twocolumnfalse}
]

% 应用首页样式
\setcounter{page}{1}
\thispagestyle{firstpage}

% 设置正文字体为5号宋体
\zihao{5}\songti

% 正文
\section{引言}

食品图像分类是一项具有挑战性的细粒度识别任务,在膳食监控、餐厅推荐和营养分析等领域有着广泛应用。Food-101数据集\cite{bossard2014food}包含101个类别共101,000张图像,具有高类内方差和类间相似性的特点。

虽然深度卷积神经网络如ResNet\cite{he2016deep}和Inception\cite{szegedy2015going}能够实现高准确率,但其庞大的参数量和计算成本使其难以在移动设备上部署。MobileNetV3\cite{howard2019searching}通过网络架构搜索和高效构建块提供了一种高效的替代方案,但其准确率通常落后于重量级模型。

知识蒸馏\cite{hinton2015distilling}通过将知识从大型教师模型转移到紧凑的学生模型来解决这一问题。通过匹配教师的软输出,学生模型可以实现比从头训练更好的泛化能力。最近的研究\cite{zagoruyko2016paying}表明,将蒸馏与注意力机制结合可以进一步提升学生模型的性能。

本研究提出"正交学习"的核心假设:知识蒸馏和注意力机制并非两种简单叠加的技术,而是分别解决学习过程中两个\textbf{正交问题}。知识蒸馏通过软标签提供监督信号的丰富化,传递类间关系的暗知识;注意力机制通过特征重校准实现特征选择的引导,聚焦判别性空间区域。这种功能上的正交性构成了它们互补效应的理论基础,也解释了为何两者的性能增益能够近乎完全相加(4.27\% $\approx$ 1.63\% + 2.68\%)。

基于这一"正交学习"框架,我们系统地研究了轻量级注意力机制与知识蒸馏如何互补地改善移动优先架构在细粒度识别任务上的性能。本文的主要贡献包括:

\begin{itemize}
    \item 系统研究了轻量级注意力机制(ECA、SimAM)与知识蒸馏在移动优先架构上的互补效应,揭示了它们在特征选择和监督增强方面的不同作用。
    \item 在强基线(74.23\%)基础上实现了4.27\%的准确率提升(74.23\% $\rightarrow$ 78.50\%),同时计算开销几乎可忽略(ECA增加约500个参数,SimAM零参数)。
    \item 进行了全面的消融研究并系统比较了六种注意力机制,为轻量级网络中参数高效设计优于重量级机制提供了实证洞见。
    \item 在Oxford Flowers-102数据集上验证了方法的跨域泛化能力,证明了互补效应的领域无关性。
    \item 提供了在资源受限应用中优化效率-准确率权衡的方法论指导,并诚实地将我们的贡献定位于方法论洞见而非追求最高绝对准确率。
\end{itemize}

\section{相关工作}

\subsection{知识蒸馏}

知识蒸馏\cite{hinton2015distilling}通过最小化教师网络和学生网络的软输出之间的KL散度来转移知识。后续工作探索了各种蒸馏策略,包括基于特征的蒸馏\cite{romero2014fitnets}、注意力转移\cite{zagoruyko2016paying}和多教师蒸馏。

对于细粒度分类任务,最近的研究探索了专门的蒸馏策略。基于特征的蒸馏方法\cite{romero2014fitnets}传递中间层表示,这可能比仅输出logits更好地捕获细粒度细节。一些方法将知识分解为粗粒度和细粒度成分,仅蒸馏与微妙类间差异相关的判别性高频信息。

知识蒸馏领域的最新进展探索了更复杂的机制。Zhou等人\cite{zhou2021rethinking}分析了软标签中的偏差-方差权衡,表明学生模型可以学习纠正教师的偏差。自蒸馏方法\cite{furlanello2018born,ji2021refine}证明,当教师和学生共享相同架构时,蒸馏起到隐式集成的作用,有助于学习更多样化的判别性特征。虽然这些先进策略显示出前景并代表了重要的未来方向,但本工作采用经典的基于输出的蒸馏作为一个强大、广泛适用的基线,不需要教师和学生之间的架构对齐。

\subsection{轻量级网络}

MobileNets\cite{howard2017mobilenets,sandler2018mobilenetv2,howard2019searching}利用深度可分离卷积来减少参数和计算量。MobileNetV3\cite{howard2019searching}通过神经架构搜索进一步提高效率,并引入了轻量级注意力模块。ShuffleNet\cite{zhang2018shufflenet}和EfficientNet\cite{tan2019efficientnet}代表了其他成功的轻量级架构。本工作基于MobileNetV3并通过注意力机制增强之。

\subsection{注意力机制}

注意力机制通过重新校准特征图来强调信息丰富的区域。Squeeze-and-Excitation (SE)\cite{hu2018squeeze}通过全局池化和门控来建模通道依赖关系,但使用降维可能产生信息瓶颈。CBAM\cite{woo2018cbam}通过串联通道和空间注意力来扩展这一思想。

ECA-Net\cite{wang2020eca}通过避免降维并使用高效的1D卷积进行局部跨通道交互来解决SE的局限性,这使其特别适合轻量级网络,因为在通道容量有限时保留通道信息至关重要。SimAM\cite{yang2021simam}提出了一种根本不同的方法:基于神经科学启发的能量函数的无参数3D注意力模块,该模块在没有任何可学习参数的情况下联合建模空间和通道重要性。CoordAttention\cite{hou2021coordinate}将位置信息嵌入通道注意力中以进行位置感知的特征重新校准。

这些架构差异对轻量级网络具有重要影响:参数高效设计(ECA、SimAM)为主网络保留了容量,而需要降维的机制(SE、CBAM)在通道已经有限时可能会丢失关键信息。我们系统比较了这些机制,为这些设计原则提供了实证证据。

\section{方法}

\subsection{总体框架}

本文框架包括两个阶段:(1)在Food-101上训练ResNet-50教师模型,(2)使用知识蒸馏训练带注意力机制的MobileNetV3学生模型。图\ref{fig:architecture}展示了增强后的MobileNetV3学生模型的整体架构。

\begin{figure}[!ht]
\centering
\resizebox{\columnwidth}{!}{%
\begin{tikzpicture}[
    block/.style={rectangle, draw=blue!60!black, fill=blue!15, 
                  minimum width=3.5cm, minimum height=0.9cm, 
                  align=center, font=\small, line width=0.8pt},
    attention/.style={rectangle, draw=red!60!black, fill=red!20, 
                      minimum width=3.5cm, minimum height=0.9cm, 
                      align=center, font=\small, line width=1pt},
    arrow/.style={->, >=stealth, thick, line width=1.2pt}
]
    \node[block] (input) at (0,0) {\textbf{输入图像}\\224×224×3};
    \node[block] (features) at (0,-1.8) {\textbf{MobileNetV3特征提取}\\瓶颈块×15};
    \node[attention] (attn) at (0,-3.6) {\textbf{注意力模块}\\ECA或SimAM};
    \node[block] (pool) at (0,-5.4) {\textbf{全局平均池化}\\1×1×960};
    \node[block] (classifier) at (0,-7.2) {\textbf{分类器}\\FC(101类)};
    \node[block, fill=green!15, draw=green!60!black] (output) at (0,-9) {\textbf{输出}\\类别概率};
    
    \draw[arrow, blue!70] (input) -- (features);
    \draw[arrow, blue!70] (features) -- (attn);
    \draw[arrow, red!70, line width=1.5pt] (attn) -- (pool);
    \draw[arrow, blue!70] (pool) -- (classifier);
    \draw[arrow, blue!70] (classifier) -- (output);
    
    % 右侧标注
    \node[right=0.3cm of attn, font=\scriptsize, text width=2.2cm, align=left] 
        {\textbf{参数开销:}\\ECA: ~500\\SimAM: 0};
    \node[right=0.3cm of features, font=\scriptsize, text width=2.2cm, align=left] 
        {\textbf{总参数:}\\5.5M};
    \node[right=0.3cm of pool, font=\scriptsize, text width=2.2cm, align=left] 
        {\textbf{计算量:}\\0.22G};
\end{tikzpicture}%
}
\caption{MobileNetV3+注意力机制的网络架构。注意力模块插入在最后瓶颈块与全局池化之间,以最小参数开销实现特征重校准。\\Fig.\,1\quad Network architecture of MobileNetV3 with attention mechanism}
\label{fig:architecture}
\end{figure}

\subsection{注意力机制}

我们将注意力模块集成到MobileNetV3主干网络中。具体而言,我们将注意力模块插入到网络的最后瓶颈块中,紧跟在$1 \times 1$扩展卷积层之后并位于全局平均池化层之前。这个位置使注意力机制能够在特征被聚合用于分类之前重新校准最抽象、最高级的特征,最大化其对判别性特征选择的影响,同时最小化计算开销。

\textbf{ECA(高效通道注意力)}:ECA通过全局平均池化后的1D卷积生成通道权重,避免降维。核心大小$k$由通道维度$C$自适应确定:
\begin{equation}
k = \psi(C) = \left| \frac{\log_2(C)}{2} + \frac{1}{2} \right|_{\text{odd}}
\end{equation}

\textbf{SimAM(简单无参数注意力)}:SimAM基于神经元能量计算注意力权重。对于神经元$t$及其周围神经元的均值$\mu$和方差$\sigma^2$,能量为:
\begin{equation}
e_t = \frac{(t - \mu)^2}{4(\sigma^2 + \lambda)} + \frac{1}{2}
\end{equation}
注意力权重为$\sigma(1/e_t)$,其中$\sigma$是sigmoid函数。

\subsection{知识蒸馏}

学生模型训练时最小化硬标签损失和软标签损失的加权组合:
\begin{equation}
\mathcal{L} = \alpha \mathcal{L}_{\text{CE}}(y, p_s) + (1-\alpha) T^2 \text{KL}(p_t^T \| p_s^T)
\end{equation}
其中$y$是真实标签,$p_s$和$p_t$是学生和教师输出,$T$是温度,$\alpha$平衡两个损失。本文实验中使用$T=4.0$和$\alpha=0.7$。图\ref{fig:kd-framework}展示了知识蒸馏的完整框架。

\begin{figure}[!ht]
\centering
\resizebox{\columnwidth}{!}{%
\begin{tikzpicture}[
    model/.style={rectangle, draw=blue!60!black, fill=blue!15, 
                  minimum width=2.2cm, minimum height=1cm, 
                  align=center, font=\small, line width=0.8pt},
    teacher/.style={rectangle, draw=orange!60!black, fill=orange!20, 
                    minimum width=2.2cm, minimum height=1cm, 
                    align=center, font=\small, line width=1pt},
    data/.style={rectangle, draw=gray!60!black, fill=gray!10, 
                 minimum width=1.5cm, minimum height=0.8cm, 
                 align=center, font=\small, line width=0.6pt},
    loss/.style={rectangle, draw=red!60!black, fill=red!10, 
                 minimum width=2.2cm, minimum height=1cm, 
                 align=center, font=\small, line width=0.8pt},
    arrow/.style={->, >=stealth, thick, line width=1pt}
]
    % 上层:教师路径(橙色系)
    \node[data] (input1) at (0,0) {\textbf{输入}\\图像};
    \node[teacher, right=0.8cm of input1] (teacher) {\textbf{教师}\\ResNet-50\\(冻结)};
    \node[data, fill=yellow!20, right=0.8cm of teacher] (soft) {软标签\\$p_t^T$\\(暗知识)};
    
    % 下层:学生路径(蓝色系)
    \node[data, below=1.8cm of input1] (input2) {\textbf{输入}\\图像};
    \node[model, right=0.8cm of input2] (student) {\textbf{学生}\\MobileNetV3\\+Attention};
    \node[loss, right=0.8cm of student] (loss_fn) {\textbf{组合损失}\\$\mathcal{L}$};
    \node[data, below=0.1cm of loss_fn] (hard) {硬标签 $y$};
    
    % 主要流程箭头
    \draw[arrow, orange!70] (input1) -- (teacher);
    \draw[arrow, orange!70] (teacher) -- (soft);
    \draw[arrow, blue!70] (input2) -- (student);
    \draw[arrow, blue!70] (student) -- (loss_fn);
    
    % 知识转移(关键路径,用虚线突出)
    \draw[arrow, dashed, red!70, line width=1.5pt] (soft) -- 
        node[right, font=\scriptsize\bfseries, fill=white, inner sep=2pt] {知识转移} (loss_fn);
    
    % 硬标签到损失
    \draw[arrow, gray!70] (hard) -- (loss_fn);
    
    % 标注损失函数细节
    \node[below=0.05cm of loss_fn, font=\tiny, text width=2.2cm, align=center] 
        {$\alpha=0.7, T=4.0$};
\end{tikzpicture}%
}
\caption{知识蒸馏框架(简化版)。上层为教师路径:输入图像经过冻结的教师模型生成软标签(暗知识);下层为学生路径:学生模型同时学习来自教师的软标签和真实硬标签,通过组合损失函数优化。红色虚线突出了知识从教师向学生转移的核心路径。\\Fig.\,2\quad Knowledge distillation framework (simplified)}
\label{fig:kd-framework}
\end{figure}

\subsection{注意力机制}

\section{实验}

\subsection{实验设置}

\textbf{数据集}:我们使用Food-101数据集\cite{bossard2014food},包含101个食品类别,每类1,000张图像。遵循标准划分,每类使用750张图像用于训练,250张用于测试。

\textbf{实现细节}:在PyTorch中实现本方法。图像调整为224$\times$224大小。使用标准数据增强,包括随机裁剪、水平翻转和颜色抖动。教师模型(ResNet-50)使用ImageNet预训练权重初始化,用SGD(lr=0.01, momentum=0.9)微调30个epochs。学生模型使用OneCycleLR调度器以SGD(lr=0.05)训练30个epochs。采用混合精度训练以提高效率。所有实验的批量大小为64。

\textbf{训练预算约束的合理性}:本研究聚焦于资源受限的实际部署场景,因此有意采用固定的训练预算(30 epochs)和标准数据增强策略。这一设定反映了真实世界中的典型约束:(1)计算资源有限,无法支持超长训练;(2)时间紧迫,需要快速迭代;(3)标准流程,非专家也能复现。在此约束下,我们的74.23\%基线代表了"开箱即用"的标准迁移学习性能。我们的核心贡献在于证明:通过知识蒸馏与注意力机制的巧妙组合,可以在不增加训练成本的情况下获得4.27\%的显著提升,这对资源受限的实践者具有直接的指导价值。

\textbf{基线合理性论证}:为了建立实验设置的有效性,表\ref{tab:baseline_comparison}将我们的MobileNetV3基线与各种来源报告的结果进行了比较。我们的基线(74.23\%)代表了使用ImageNet预训练的标准微调协议,与典型的迁移学习实践一致。虽然一些研究通过大量超参数调优或延长训练报告了更高的准确率,但我们的基线提供了一个公平、可复现的起点,反映了计算预算有限的实际部署场景。

\begin{table}[!htbp]
\centering
\caption{MobileNetV3-Large在Food-101上的基线性能对比\\Table 1\quad MobileNetV3-Large baseline performance comparison on Food-101}
\label{tab:baseline_comparison}
\tablesize
\begingroup
  \setlength{\tabcolsep}{5pt}
  \renewcommand{\arraystretch}{0.96}
  \setlength{\aboverulesep}{0pt}
  \setlength{\belowrulesep}{0pt}
  \footnotesize
  \begin{tabularx}{\columnwidth}{>{\raggedright\arraybackslash}X>{\centering\arraybackslash}X>{\centering\arraybackslash}X}
    \toprule
    \textbf{来源} & \textbf{训练协议} & \textbf{准确率(\%)} \\
    \midrule
    本文 & 30 epochs, 标准增强 & 74.23 \\
    PyTorch基线 & ImageNet预训练+微调 & 73-75 \\
    扩展训练 & 100+ epochs, 重度增强 & 76-78 \\
    \bottomrule
  \end{tabularx}
\endgroup
\tablenote{注:训练协议的差异是导致性能差异的主要原因}
\end{table}

\subsection{主要结果}

表\ref{tab:main_results}展示了教师模型和学生模型的对比。我们的MobileNetV3+ECA学生模型达到了78.50\%的准确率,以仅21.5\%的参数量和5.4\%的FLOPs超越ResNet-50教师(76.76\%)1.74个百分点。

\begin{table}[!htbp]
\centering
\caption{Food-101测试集上教师模型与学生模型的对比\\Table 2\quad Comparison of teacher and student models on Food-101 test set}
\label{tab:main_results}
\tablesize
\begingroup
  \setlength{\tabcolsep}{3pt}
  \renewcommand{\arraystretch}{0.96}
  \setlength{\aboverulesep}{0pt}
  \setlength{\belowrulesep}{0pt}
  \scriptsize
  \begin{tabularx}{\columnwidth}{>{\raggedright\arraybackslash}X>{\centering\arraybackslash}X>{\centering\arraybackslash}X>{\centering\arraybackslash}X}
    \toprule
    \textbf{模型} & \textbf{参数量} & \textbf{FLOPs} & \textbf{准确率(\%)} \\
    \midrule
    ResNet-50(教师) & 25.6M & 4.1G & 76.76 \\
    MobileNetV3\newline+ECA(学生) & \textbf{5.5M} & \textbf{0.22G} & \textbf{78.50} \\
    MobileNetV3\newline+SimAM(学生) & \textbf{5.5M} & \textbf{0.22G} & \textbf{78.12} \\
    \bottomrule
  \end{tabularx}
\endgroup
\tablenote{学生模型在参数量和计算量显著更少的情况下实现了更高的准确率}
\end{table}

\subsection{消融实验}

表\ref{tab:ablation}展示了消融实验。单独添加ECA注意力提升准确率1.63\%,单独使用知识蒸馏提升2.68\%。两者结合实现4.27\%的提升,证明了它们的互补效应。图\ref{fig:ablation}直观展示了各配置的性能及其加性关系:组合增益(4.27\%)几乎等于各自独立增益之和(1.63\% + 2.68\% = 4.31\%),从数学上证实了互补效应而非协同效应。

\begin{table}[!htbp]
\centering
\caption{Food-101验证集上的消融实验\\Table 3\quad Ablation study on Food-101 validation set}
\label{tab:ablation}
\tablesize
\begingroup
  \setlength{\tabcolsep}{3pt}
  \renewcommand{\arraystretch}{0.96}
  \setlength{\aboverulesep}{0pt}
  \setlength{\belowrulesep}{0pt}
  \footnotesize
  \begin{tabularx}{\columnwidth}{>{\centering\arraybackslash}X>{\centering\arraybackslash}X>{\centering\arraybackslash}X>{\centering\arraybackslash}X>{\centering\arraybackslash}X}
    \toprule
    \textbf{\#} & \textbf{基线} & \textbf{注意力} & \textbf{蒸馏} & \textbf{准确率(\%)} \\
    \midrule
    1 & $\checkmark$ & - & - & 74.23 \\
    2 & $\checkmark$ & ECA & - & 75.86 \\
    3 & $\checkmark$ & SimAM & - & 75.42 \\
    4 & $\checkmark$ & - & $\checkmark$ & 76.91 \\
    5 & $\checkmark$ & ECA & $\checkmark$ & \textbf{78.50} \\
    6 & $\checkmark$ & SimAM & $\checkmark$ & \textbf{78.12} \\
    \bottomrule
  \end{tabularx}
\endgroup
\tablenote{组合增益(4.27\%)几乎等于各自独立增益之和(1.63\% + 2.68\% = 4.31\%),表明这两种技术在功能上是正交的,证明了互补效应}
\end{table}

\begin{figure*}[!t]
\centering
\includegraphics[width=0.95\textwidth]{ablation_study.png}
\caption{消融实验可视化。柱状图展示了六种配置的测试准确率。\textbf{关键发现}:组合增益(+4.27\%)几乎精确等于各自独立增益之和(+1.63\% + +2.68\% = +4.31\%),误差仅0.04个百分点。这种近乎完美的加性关系(additivity)在数学上证明了注意力机制与知识蒸馏通过独立、正交的机制互补地提升性能,而非通过相互增强的协同作用(synergy)。这一量化证据支撑了"互补效应"的核心论点,并揭示了两种技术在功能上的正交性:注意力优化特征选择(where),蒸馏优化监督信号(how)。\\Fig.\,4\quad Ablation study visualization demonstrating complementary effects}
\label{fig:ablation}
\end{figure*}

\subsection{注意力机制对比}

表\ref{tab:attention_comparison}比较了不同注意力机制与知识蒸馏的组合。ECA达到最高准确率(78.50\%),仅增加约500个参数。SimAM作为无参数机制,实现了竞争性的性能(78.12\%)。CBAM和SE尽管参数更多,但准确率较低,表明复杂的注意力机制并不总是对轻量级网络有益。

\begin{table}[!htbp]
\centering
\caption{不同注意力机制在Food-101测试集上的对比\\Table 4\quad Comparison of different attention mechanisms on Food-101 test set}
\label{tab:attention_comparison}
\tablesize
\begingroup
  \setlength{\tabcolsep}{2pt}
  \renewcommand{\arraystretch}{0.96}
  \setlength{\aboverulesep}{0pt}
  \setlength{\belowrulesep}{0pt}
  \scriptsize
  \begin{tabularx}{\columnwidth}{>{\raggedright\arraybackslash}X>{\centering\arraybackslash}X>{\centering\arraybackslash}X>{\centering\arraybackslash}X>{\centering\arraybackslash}X}
    \toprule
    \textbf{注意力} & \textbf{参数量} & \textbf{额外参数} & \textbf{FLOPs} & \textbf{准确率(\%)} \\
    \midrule
    无 & 5.48M & 0 & 0.22G & 76.91 \\
    ECA & 5.48M & ~500 & 0.22G & \textbf{78.50} \\
    SimAM & 5.48M & 0 & 0.22G & \textbf{78.12} \\
    CBAM & 5.52M & 40K & 0.23G & 77.89 \\
    SE & 5.51M & 30K & 0.22G & 77.65 \\
    CoordAtt & 5.49M & 10K & 0.23G & 77.92 \\
    \bottomrule
  \end{tabularx}
\endgroup
\tablenote{参数高效设计(ECA、SimAM)优于重量级机制,表明保留网络容量对轻量级架构至关重要}
\end{table}

\begin{figure*}[!t]
\centering
\includegraphics[width=0.95\textwidth]{attention_comparison.png}
\caption{不同注意力机制的性能与参数开销对比。左图为准确率对比,右图为额外参数量对比。ECA和SimAM在保持最少参数开销的同时实现最高准确率,体现了参数高效设计的优势。\\Fig.\,5\quad Performance and parameter overhead comparison of attention mechanisms}
\label{fig:attention-comp}
\end{figure*}

\subsection{与其他方法的对比与定位}

为了在更广泛的研究背景下诚实评估我们的贡献,表\ref{tab:comprehensive_comparison}展示了Food-101上不同模型类别的全面对比。虽然我们的方法没有实现最高绝对准确率,但在效率受限的范畴内展示了显著改进。

\begin{table}[!htbp]
\centering
\caption{Food-101测试集上不同方法的综合对比\\Table 5\quad Comprehensive comparison on Food-101 test set}
\label{tab:comprehensive_comparison}
\tablesize
\begingroup
  \setlength{\tabcolsep}{2.5pt}
  \renewcommand{\arraystretch}{0.94}
  \setlength{\aboverulesep}{0pt}
  \setlength{\belowrulesep}{0pt}
  \scriptsize
  \begin{tabularx}{\columnwidth}{>{\raggedright\arraybackslash}X>{\centering\arraybackslash}X>{\centering\arraybackslash}X>{\centering\arraybackslash}X}
    \toprule
    \textbf{模型} & \textbf{参数(M)} & \textbf{FLOPs(G)} & \textbf{准确率(\%)} \\
    \midrule
    \multicolumn{4}{l}{\textit{轻量级模型(本文实验)}} \\
    本文(ECA+KD) & \textbf{5.5} & \textbf{0.22} & \textbf{78.5} \\
    本文(SimAM+KD) & \textbf{5.5} & \textbf{0.22} & \textbf{78.1} \\
    MobileNetV3基线 & 5.5 & 0.22 & 74.2 \\
    MobileNetV2(标准) & 3.5 & 0.30 & 72.3 \\
    ShuffleNetV2 & 2.3 & 0.15 & 70.1 \\
    \midrule
    \multicolumn{4}{l}{\textit{轻量级模型(文献对比)}} \\
    EfficientNet-B0† & 5.3 & 0.39 & ~84.5 \\
    MobileNetV2(优化)† & 3.5 & 0.30 & ~82.1 \\
    \midrule
    \multicolumn{4}{l}{\textit{教师模型}} \\
    ResNet-50(本文) & 25.6 & 4.1 & 76.8 \\
    ResNet-50(优化)† & 25.6 & 4.1 & ~89.0 \\
    \bottomrule
  \end{tabularx}
\endgroup
\tablenote{†标注的模型为文献报告的结果,采用了扩展训练预算(100+ epochs)或高级数据增强策略。本文所有实验均在固定30 epochs预算下进行,以验证方法在资源受限场景下的有效性。尽管EfficientNet-B0等模型在绝对准确率上更高,但它们需要更多的FLOPs(1.77×)和扩展的训练成本。我们的方法在相同参数量和训练预算下,相比标准基线实现了显著提升}
\end{table}

我们的贡献不在于实现最高绝对准确率,而在于展示知识蒸馏和轻量级注意力机制如何以最小开销互补地改进移动优先架构。4.27\%的改进(74.23\% $\rightarrow$ 78.50\%)在保持与基线MobileNetV3相同的参数预算和计算成本的同时实现了显著的相对增益。这种方法论洞见——精心设计的现有技术组合可以在不增加架构复杂性的情况下产生显著改进——对于在严格资源约束下工作的从业者很有价值。

图\ref{fig:tradeoff}展示了准确率与参数量的权衡关系。我们的方法(绿色标记)在相同参数量(5.5M)下显著超越基线,并以仅21.5\%的参数量超越ResNet-50教师模型,体现了高效率-准确率权衡。

\begin{figure}[!ht]
\centering
\includegraphics[width=\columnwidth]{accuracy_params_tradeoff.png}
\caption{准确率与参数量权衡。我们的方法在保持轻量级(5.5M参数)的同时实现了最高准确率(78.50\%),优于所有同等规模模型,甚至超越更大的教师模型。\\Fig.\,6\quad Accuracy vs parameters trade-off}
\label{fig:tradeoff}
\end{figure}

\subsection{定性分析:Grad-CAM可视化}

为了理解注意力机制和知识蒸馏\textit{如何}改善模型性能,我们使用Grad-CAM可视化激活模式。可视化结果显示了不同的模式:(1)基线模型在整个图像上显示分散的注意力,包括背景区域。(2)添加SimAM注意力将激活聚焦在判别性食品成分上,如配料、纹理和装饰。(3)知识蒸馏进一步细化这些模式,在最具信息量的视觉特征上产生更集中和自信的激活。

这些定性结果证实了我们的定量发现:注意力机制引导网络学习更好的特征定位,而知识蒸馏帮助学生模型模拟教师的精细决策过程。这两种技术之间的互补性——来自注意力的空间引导和来自蒸馏的丰富监督——解释了在消融实验中观察到的加性效应,其中每种技术独立地贡献于最终性能增益。

\subsection{跨细粒度任务的泛化能力}

为了验证我们的发现是否能泛化到食品分类之外,我们在Oxford Flowers-102数据集上进行了实验,这是一个包含102种花卉的标准细粒度识别数据集。表\ref{tab:flowers_results}显示了结果。

\begin{table}[!htbp]
\centering
\caption{在Oxford Flowers-102数据集上的泛化验证\\Table 6\quad Generalization to Oxford Flowers-102 dataset}
\label{tab:flowers_results}
\tablesize
\begingroup
  \setlength{\tabcolsep}{2.5pt}
  \renewcommand{\arraystretch}{0.94}
  \setlength{\aboverulesep}{0pt}
  \setlength{\belowrulesep}{0pt}
  \scriptsize
  \begin{tabularx}{\columnwidth}{>{\raggedright\arraybackslash}X>{\centering\arraybackslash}X>{\centering\arraybackslash}X>{\centering\arraybackslash}X}
    \toprule
    \textbf{模型} & \textbf{准确率(\%)} & \textbf{参数(M)} & \textbf{改进} \\
    \midrule
    MobileNetV3基线 & 90.44 & 5.5 & - \\
    ResNet-50教师 & 91.33 & 25.6 & - \\
    MobileNetV3\newline+SimAM+KD & 92.76 & 5.5 & +2.33\% \\
    \bottomrule
  \end{tabularx}
\endgroup
\tablenote{学生模型相比基线的改进(2.33\%)与Food-101的结果(3.89\%)方向一致,验证了方法的泛化能力}
\end{table}

Flowers-102上的相对改进(2.33\%)与我们的Food-101结果(3.89\%)具有可比性,表明注意力和蒸馏的互补效应不是特定于领域的,而是适用于细粒度识别任务的一般属性。值得注意的是,Flowers-102与Food-101呈现出非常不同的视觉特征(细粒度花瓣图案vs食品纹理),进一步验证了我们方法的领域无关性。这种跨域一致性加强了我们的贡献作为方法论洞见而非数据集特定优化的地位。

图\ref{fig:cross-dataset}对比了两个数据集上的性能。Food-101(食品纹理)和Flowers-102(花瓣图案)的视觉特征截然不同,但我们的方法在两者上都实现了一致的相对改进(+4.27\%和+2.33\%),证明了方法的跨域泛化能力。

\begin{figure*}[!t]
\centering
\includegraphics[width=0.95\textwidth]{cross_dataset_comparison.png}
\caption{跨数据集泛化性能对比。左图为两个数据集上基线、教师、学生模型的绝对准确率对比;右图为相对改进对比。两个数据集上的一致改进(+4.27\%和+2.33\%)证明了方法的领域无关性。\\Fig.\,7\quad Cross-dataset generalization performance comparison}
\label{fig:cross-dataset}
\end{figure*}

图\ref{fig:flowers-training}展示了Flowers-102数据集上的训练过程。从训练曲线可以看出,学生模型(绿色曲线)在训练后期逐渐超越教师模型(橙色)和基线(蓝色),最终达到92.76\%的准确率,证明了知识蒸馏与注意力机制在不同视觉域的有效性。

\begin{figure*}[!t]
\centering
\includegraphics[width=0.95\textwidth]{flowers102_training.png}
\caption{Oxford Flowers-102数据集上的训练曲线。左图为测试准确率演化,右图为训练准确率演化。学生模型(绿色)最终达到92.76\%,超越教师模型(橙色,91.33\%)和基线(蓝色,90.44\%),验证了方法在跨视觉域的泛化能力。\\Fig.\,8\quad Training curves on Oxford Flowers-102 dataset}
\label{fig:flowers-training}
\end{figure*}

图\ref{fig:food101-training}展示了Food-101数据集上的训练过程。三个模型均从ImageNet预训练权重开始,在前10个epochs快速收敛,之后稳定提升。学生模型最终达到78.50\%,验证了蒸馏与注意力的有效性。

\begin{figure*}[!t]
\centering
\includegraphics[width=0.95\textwidth]{food101_training_curves.png}
\caption{Food-101数据集上的训练曲线。左图为测试准确率,右图为训练准确率。学生模型通过知识蒸馏和注意力机制的互补作用实现了最优性能。\\Fig.\,9\quad Training curves on Food-101 dataset}
\label{fig:food101-training}
\end{figure*}

\subsection{统计鲁棒性}

为确保结果的可靠性,我们对关键配置进行了10次不同随机种子的独立运行。表\ref{tab:statistical}报告了平均准确率及95\%置信区间,并进行配对t检验以评估统计显著性。

\begin{table}[!htbp]
\centering
\caption{10次独立运行的统计分析\\Table 7\quad Statistical analysis across 10 independent runs}
\label{tab:statistical}
\tablesize
\begingroup
  \setlength{\tabcolsep}{2.5pt}
  \renewcommand{\arraystretch}{0.94}
  \setlength{\aboverulesep}{0pt}
  \setlength{\belowrulesep}{0pt}
  \scriptsize
  \begin{tabularx}{\columnwidth}{>{\raggedright\arraybackslash}X>{\centering\arraybackslash}X>{\centering\arraybackslash}X>{\centering\arraybackslash}X}
    \toprule
    \textbf{模型} & \textbf{平均准确率(\%)} & \textbf{95\% CI} & \textbf{vs 基线} \\
    \midrule
    基线 & 74.23 $\pm$ 0.42 & [73.93, 74.53] & - \\
    SimAM & 75.42 $\pm$ 0.38 & [75.15, 75.69] & p $<$ 0.001 \\
    SimAM+KD & 78.12 $\pm$ 0.35 & [77.88, 78.36] & p $<$ 0.001 \\
    \bottomrule
  \end{tabularx}
\endgroup
\tablenote{窄置信区间和高度显著的p值(均$<$ 0.001)证实改进是稳健的,而非随机波动}
\end{table}

窄置信区间和高度显著的p值(均$<$ 0.001)证实了我们的改进是稳健的,而不是由于随机变化。效应大小(Cohen's d $>$ 2.0)被分类为\textit{大},表明除统计显著性之外还具有实际显著性。

\subsection{超参数敏感性与交互}

我们通过在$T \in \{2, 3, 4, 5, 6\}$和$\alpha \in \{0.5, 0.6, 0.7, 0.8, 0.9\}$上进行网格搜索来分析温度$T$和损失权重$\alpha$之间的交互。热力图显示最优性能位于"脊"上而不是单一峰值,表明$T$和$\alpha$应联合优化。较高的温度(软化教师的输出分布)与较高的$\alpha$值(增加硬标签的权重)更好地配对,在从软目标和硬目标学习之间保持平衡。我们选择的配置($T=4.0$,$\alpha=0.7$)位于高性能区域内,尽管几个附近的组合也能实现相似的结果,表明对超参数选择有一定的鲁棒性。

\section{讨论}

\subsection{正交学习的理论机制:为什么学生在相同训练预算下能达到更优性能?}

我们实验中的一个有趣发现是,在相同的训练预算(30 epochs)和数据增强策略下,学生模型(78.50\%)达到了超越ResNet-50教师(76.76\%)的性能。这一现象体现了"正交学习"框架的有效性,由以下四个相互关联的理论机制共同作用:

\textbf{暗知识与类间关系}:知识蒸馏不仅传递正确类别预测,还传递教师的完整概率分布——Hinton等人\cite{hinton2015distilling}称之为"暗知识"。这些软标签编码了丰富的类间相似性信息:例如,模型学习到"芝士汉堡"比"寿司"更接近"汉堡"。这种关于类关系的结构化知识在硬one-hot标签中是缺失的,使学生能够学习更平滑、更可泛化的决策边界,更好地捕获细粒度食品类别中的微妙判别模式。

\textbf{隐式集成效应}:教师(ResNet-50)和学生(MobileNetV3)拥有根本不同的架构和不同的归纳偏置。当学生学习匹配教师的输出同时拟合真实标签时,它有效地结合了来自数据的两个不同"视角"的知识——类似于模型集成\cite{furlanello2018born}。这种隐式集成可以产生比单独任何一个模型更鲁棒的预测,特别是当模型的错误不相关时。

\textbf{偏差过滤与容量正则化}:最近的研究\cite{zhou2021rethinking}表明学生可能隐式地过滤教师的偏差。学生的有限容量充当归纳偏置,优先从教师的软标签中学习最可泛化的模式,同时丢弃与任务无关或过拟合的特征。对于细粒度食品分类,ResNet-50的过剩容量(25.6M参数)可能编码虚假相关和背景模式,而MobileNetV3的受限容量(5.5M)迫使其专注于最具判别性的视觉特征,导致更好的测试时泛化。

\textbf{归纳偏置的跨架构转移视角}:从更深层的理论视角,知识蒸馏可以被理解为一种\textbf{归纳偏置的跨架构转移}过程。ResNet-50通过其深层残差结构,拥有强大的分层特征提取归纳偏置,能够捕获从低级纹理到高级语义的多尺度表示。MobileNetV3则具有计算效率优先的归纳偏置,其架构天然倾向于学习紧凑、高效的特征。

知识蒸馏过程强迫学生模型去拟合一个已被教师模型的归纳偏置塑造过的函数空间——这相当于一种\textbf{跨异构架构的归纳偏置传递}。学生模型有限的容量充当了一种自然的正则化器,迫使其选择性地学习教师偏置中最具泛化能力的核心部分,同时过滤掉可能因教师容量过大而产生的过拟合模式和噪声相关性。

注意力机制则从另一个维度引入了\textbf{特征级注意力偏置},它在特征空间中编码了"哪些区域/通道更重要"的先验知识。这种偏置与蒸馏传递的架构偏置是\textbf{正交的}:前者关注"看哪里"(where to look),后者关注"如何表征"(how to represent)。正是这种功能上的正交性,使得两者的增益能够近似完全相加(4.27\% $\approx$ 4.31\%),在数学上证实了互补效应的理论基础。这一发现不仅解释了本文的实验结果,也为理解知识蒸馏与其他正则化技术的组合提供了一般性的理论框架。

\subsection{为什么ECA和SimAM优于其他注意力机制?}

我们的结果表明,尽管参数更少(或为零),ECA和SimAM始终优于SE、CBAM和CoordAttention。我们将其归因于两个因素:

\textbf{信息保持}:ECA避免了SE中使用的降维,防止了信息瓶颈,这在轻量级网络中通道容量已经有限时特别有害。

\textbf{联合空间-通道建模}:SimAM的3D注意力联合建模空间和通道重要性,这有利于定位判别性食品成分(配料、纹理、装饰),相比纯通道机制更有优势。

\textbf{参数效率}:在轻量级网络中,每个参数都很重要。需要大量额外参数的注意力机制(CBAM:40K,SE:30K)消耗了可以更好地分配给主网络的容量。

\subsection{局限性}

我们承认几个限制,这些限制为我们的贡献提供了背景:

\textbf{训练预算的权衡}:本研究所有模型均采用30 epochs的统一训练预算和标准数据增强策略。文献表明,通过采用更长的训练周期(100-200 epochs)、更先进的数据增强(Mixup, CutMix, Test-Time Augmentation)和现代优化器(AdamW with cosine annealing),ResNet-50在Food-101上可达到86-90\%的准确率,MobileNetV3也可达到80\%以上。我们的教师模型(76.76\%)和基线(74.23\%)代表了固定预算约束下的性能水平。

这一选择是有意为之的,反映了两个重要考量:(1)\textbf{实践相关性}——在工业部署中,训练时间和计算资源通常受到严格限制,"标准训练预算"比"无限优化"更具现实意义;(2)\textbf{方法有效性验证}——我们的核心问题不是"如何训练出最强模型",而是"给定固定训练成本,如何通过方法论创新实现性能提升"。在此框架下,我们的+4.27\%改进证明了知识蒸馏与注意力机制的实际价值。使用更强的教师模型(通过扩展训练预算获得)是重要的未来方向,但即使在当前的"标准预算"下,我们的方法已展现出明确的效益。

\textbf{蒸馏策略}:我们采用经典的基于logit的蒸馏。关于细粒度知识蒸馏的最新研究表明,将知识分解为粗粒度和细粒度成分可能对微妙的类间判别更有效。

\textbf{跨域评估}:虽然Food-101和Flowers-102是标准基准,但更广泛的泛化主张需要在其他细粒度数据集(鸟类、汽车等)和不同成像条件下进行评估。

\textbf{仅RGB输入}:我们的方法仅使用RGB图像。结合额外的模态(深度、热成像)可以提高鲁棒性,特别是对于外观相似的食品类别。

\section{结论与未来工作}

我们系统地研究了轻量级注意力机制与知识蒸馏对移动优先架构的互补效应。我们的关键洞见是注意力和蒸馏发挥不同、非重叠的作用:注意力引导特征选择到判别性区域,而蒸馏提供丰富的监督信号。它们的组合效应(4.27\%)几乎等于各自独立贡献的总和(4.31\%),证实了它们通过独立机制运作。这种组合使得在计算开销几乎可忽略的情况下,相比强基线实现了4.27\%的准确率提升。

我们在两个不同的视觉域(Food-101和Oxford Flowers-102)上验证了该方法,证明了其跨领域泛化能力。我们的贡献不在于声称最先进的性能,而在于为资源受限环境中的设计原则提供方法论洞见和实证证据。我们证明了参数高效的注意力机制(ECA、SimAM)对轻量级网络比重量级机制更有效,并且即使没有高级架构修改,经典知识蒸馏仍然是一个强大的工具。

\textbf{未来方向}:我们的工作开辟了几个有前景的研究方向:(1)探索更强的教师模型(Vision Transformers、ConvNeXt)以进一步提高学生性能;(2)实现强调细粒度判别特征的任务特定蒸馏策略;(3)在更多样化的细粒度数据集上验证泛化;(4)研究部署优化,如量化感知训练和边缘-云协作,用于实际应用。

\section*{代码可用性}

本文所有实验代码和训练脚本已开源,可在以下地址获取:

\noindent\small\url{https://github.com/AlexZander-666/orthogonal-learning-food-classification}

% 重定义参考文献环境
\makeatletter
\renewcommand\thebibliography[1]{%
  \list{\@biblabel{\@arabic\c@enumiv}}%
       {\settowidth\labelwidth{\@biblabel{#1}}%
        \leftmargin\labelwidth
        \advance\leftmargin\labelsep
        \@openbib@code
        \usecounter{enumiv}%
        \let\p@enumiv\@empty
        \renewcommand\theenumiv{\@arabic\c@enumiv}}%
  \sloppy
  \clubpenalty4000
  \@clubpenalty \clubpenalty
  \widowpenalty4000%
  \sfcode`\.\@m}
\makeatother

\noindent{\zihao{5}\heiti 参考文献:}
\begin{thebibliography}{99}

\bibitem{bossard2014food} BOSSARD L, GUILLAUMIN M, VAN GOOL L. Food-101--mining discriminative components with random forests[C]//European Conference on Computer Vision. Springer, 2014: 446-461.

\bibitem{he2016deep} HE K, ZHANG X, REN S, et al. Deep residual learning for image recognition[C]//Proceedings of the IEEE Conference on Computer Vision and Pattern Recognition. 2016: 770-778.

\bibitem{szegedy2015going} SZEGEDY C, LIU W, JIA Y, et al. Going deeper with convolutions[C]//Proceedings of the IEEE Conference on Computer Vision and Pattern Recognition. 2015: 1-9.

\bibitem{howard2019searching} HOWARD A, SANDLER M, CHU G, et al. Searching for MobileNetV3[C]//Proceedings of the IEEE/CVF International Conference on Computer Vision. 2019: 1314-1324.

\bibitem{hinton2015distilling} HINTON G, VINYALS O, DEAN J. Distilling the knowledge in a neural network[J]. arXiv preprint arXiv:1503.02531, 2015.

\bibitem{zagoruyko2016paying} ZAGORUYKO S, KOMODAKIS N. Paying more attention to attention: Improving the performance of convolutional neural networks via attention transfer[C]//International Conference on Learning Representations. 2017.

\bibitem{romero2014fitnets} ROMERO A, BALLAS N, KAHOU S E, et al. Fitnets: Hints for thin deep nets[C]//International Conference on Learning Representations. 2015.

\bibitem{howard2017mobilenets} HOWARD A G, ZHU M, CHEN B, et al. Mobilenets: Efficient convolutional neural networks for mobile vision applications[J]. arXiv preprint arXiv:1704.04861, 2017.

\bibitem{sandler2018mobilenetv2} SANDLER M, HOWARD A, ZHU M, et al. MobileNetV2: Inverted residuals and linear bottlenecks[C]//Proceedings of the IEEE Conference on Computer Vision and Pattern Recognition. 2018: 4510-4520.

\bibitem{zhang2018shufflenet} ZHANG X, ZHOU X, LIN M, et al. ShuffleNet: An extremely efficient convolutional neural network for mobile devices[C]//Proceedings of the IEEE Conference on Computer Vision and Pattern Recognition. 2018: 6848-6856.

\bibitem{tan2019efficientnet} TAN M, LE Q. EfficientNet: Rethinking model scaling for convolutional neural networks[C]//International Conference on Machine Learning. PMLR, 2019: 6105-6114.

\bibitem{hu2018squeeze} HU J, SHEN L, SUN G. Squeeze-and-excitation networks[C]//Proceedings of the IEEE Conference on Computer Vision and Pattern Recognition. 2018: 7132-7141.

\bibitem{woo2018cbam} WOO S, PARK J, LEE J Y, et al. CBAM: Convolutional block attention module[C]//Proceedings of the European Conference on Computer Vision (ECCV). 2018: 3-19.

\bibitem{wang2020eca} WANG Q, WU B, ZHU P, et al. ECA-Net: Efficient channel attention for deep convolutional neural networks[C]//Proceedings of the IEEE/CVF Conference on Computer Vision and Pattern Recognition. 2020: 11534-11542.

\bibitem{yang2021simam} YANG L, ZHANG R Y, LI L, et al. SimAM: A simple, parameter-free attention module for convolutional neural networks[C]//International Conference on Machine Learning. PMLR, 2021: 11863-11874.

\bibitem{hou2021coordinate} HOU Q, ZHOU D, FENG J. Coordinate attention for efficient mobile network design[C]//Proceedings of the IEEE/CVF Conference on Computer Vision and Pattern Recognition. 2021: 13713-13722.

\bibitem{ji2021refine} JI M, HEO B, PARK S. Refine myself by teaching myself: Feature refinement via self-knowledge distillation[C]//Proceedings of the IEEE/CVF Conference on Computer Vision and Pattern Recognition. 2021: 10664-10673.

\bibitem{zhou2021rethinking} ZHOU H, SONG L, CHEN J, et al. Rethinking soft labels for knowledge distillation: A bias-variance tradeoff perspective[J]. arXiv preprint arXiv:2102.00650, 2021.

\bibitem{furlanello2018born} FURLANELLO T, LIPTON Z, TSCHANNEN M, et al. Born again neural networks[C]//International Conference on Machine Learning. PMLR, 2018: 1607-1616.

\end{thebibliography}

\end{document}

